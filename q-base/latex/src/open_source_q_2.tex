\documentclass[letterpaper]{article}
\usepackage{times}
\usepackage{helvet}
\usepackage{courier}
\usepackage{fancyheadings}
\usepackage{hyperref}
\usepackage{ulem}
\pagestyle{fancy}
\usepackage{pmc}
\usepackage{graphicx}
\setlength\textwidth{6.5in}
\setlength\textheight{8.5in}

%%%%%%%%%%%%%%%%%%%%%%%%%%%%%%%%%%%%%%%%%%%%%%%%%%%%%%%%%%%%%%%%%%%%%%%%%%%%%%%%%%
% Special for e-unibus doc commands

\newcommand{\ForLater}{
\begin{center}
{\bf NOT FOR CURRENT VERSION}
\end{center}
}
\newcommand{\TBC}{\framebox{\textbf{TO BE COMPLETED}}}
\newcommand{\DISCUSS}{\Ovalbox {\bf \textcolor{red}{FOR DISCUSSION}}}
\newcommand{\Input}{\framebox{\textsf{in}}}
\newcommand{\Output}{\framebox{\textsf{out}}}
\newcommand{\debug}[1]{\textbf{debug start} #1 \textbf{debug finish}}
\newcommand{\inx}[1]{\emph{#1}}
\newtheorem{notation}{Notation}
\newtheorem{definition}{Definition}
\newtheorem{problem_statement}{Problem Statement}
\newtheorem{invariant}{Invariant}
\newtheorem{resource_string}{Resource String}
\newtheorem{testcase}{Test Case}
\newtheorem{specification}{Specification}
\newtheorem{caution}{Caution}
\newtheorem{prereq}{Pre-requisite}
\newtheorem{action}{Action}
\newtheorem{query}{Query}
\newcommand{\be}{\begin{enumerate}}
\newcommand{\ee}{\end{enumerate}}
\newcommand{\bi}{\begin{itemize}}
\newcommand{\ei}{\end{itemize}}
\newcommand{\bv}{\begin{verbatim}}
\newcommand{\ev}{\end{verbatim}}
\newcommand{\bd}{\begin{description}}
\newcommand{\ed}{\end{description}}
\newcommand{\bpre}{\begin{prereq}}
\newcommand{\epre}{\end{prereq}}
\newcommand{\bact}{\begin{action}}
\newcommand{\eact}{\end{action}}
\newcommand{\bs}{\begin{specification}}
\newcommand{\es}{\end{specification}}
\newcommand{\btc}{\begin{testcase}}
\newcommand{\etc}{\end{testcase}}
\newcommand{\bc}{\begin{caution}}
\newcommand{\ec}{\end{caution}}
\newcommand{\la}{\leftarrow}
\newcommand{\IpArgs}{\subsection{Input Arguments}}
\newcommand{\PreReqs}{\subsection{Pre-requisities}}
\newcommand{\Actions}{\subsection{Actions}}
\newcommand{\Coverage}{{\bf To test coverage.}}

%%%%%%%%%%%%%%%%%%%%%%%%%%%%%%%%%%%%%%%%%%%%%%%%%%%%%%%%%%%%%%%%%%%%%%%%%%%


\newtheorem{theorem}{Theorem}[section]
\newtheorem{lemma}[theorem]{Lemma}
\newtheorem{proposition}[theorem]{Proposition}
\newtheorem{corollary}[theorem]{Corollary}

\newenvironment{proof}[1][Proof]{\begin{trivlist}
\item[\hskip \labelsep {\bfseries #1}]}{\end{trivlist}}
\newenvironment{intuition}[1][Intuition]{\begin{trivlist}
\item[\hskip \labelsep {\bfseries #1}]}{\end{trivlist}}
%% \newenvironment{definition}[1][Definition]{\begin{trivlist}
%% \item[\hskip \labelsep {\bfseries #1}]}{\end{trivlist}}
\newenvironment{example}[1][Example]{\begin{trivlist}
\item[\hskip \labelsep {\bfseries #1}]}{\end{trivlist}}
\newenvironment{remark}[1][Remark]{\begin{trivlist}
\item[\hskip \labelsep {\bfseries #1}]}{\end{trivlist}}

\newcommand{\qed}{\nobreak \ifvmode \relax \else
      \ifdim\lastskip<1.5em \hskip-\lastskip
      \hskip1.5em plus0em minus0.5em \fi \nobreak
      \vrule height0.75em width0.5em depth0.25em\fi}

%%%%%%%%%%%%%%%%%%%%%%%%%%%%%%%%%%%%%%%%%%%%%%%%%%%%%%%%%%%%%%%%%
% \newcommand{\Alogon}{\mbox{\fontfamily{ptm}\selectfont {\large \selectfont A} \hspace{-1.2ex} {\large \selectfont L} \hspace{-2.3ex} \raisebox{0.45ex}{ {\footnotesize \selectfont O} } \hspace{-1.80ex} {\large \selectfont G} \hspace{-1.80ex} \raisebox{-0.33ex}{ {\large \selectfont O} } \hspace{-1.8ex} {\large \selectfont N}}}


\begin{document}
\title{Open-sourcing ``qtils''}
\author{ Ramesh Subramonian }
\maketitle
\thispagestyle{fancy}
\lhead{}
\chead{}
\rhead{}
\lfoot{{\small Decision Sciences Team}}
\cfoot{}
\rfoot{{\small \thepage}}

\section{Response to Jim}

Jim Brikman raised the following comments/questions. 
{\it My responses are in italics.}
\be
\item 
How does this relate to DataFu? Should it be another part of that project?

{\it The key difference is as follows. ``DataFu is a collection of user-defined
functions for working with large-scale data {\em in Hadoop and Pig}''. In
contrast, Qtils operates {\em outside} of Hadoop and Pig. 
It ingests data in the form of CSV files. These CSV files could come
from Hadoop, Teradata, \ldots. For example, in the case of GEO
standardization, the CSV files are purchased from an external vendor. In
the case of EDU standardization, we get the CSV files from the government,
Excel files from crowd-sourcing efforts using SamaSource, CrowdFlower,
\ldots The WebHDFS client makes it easy to
export data from Hadoop. The data is then operated on, transformed,
analyzed.  Potentially, the results could be put back into
Hadoop/Teradata, using the same mechanism as used to export it.
}
\item If this project goes on github, the standard is a README in
markdown format. I'd recommend converting your PDF to that and
putting it in the root folder.

{\it DONE}

\item It seems like all the source code is in a single folder. It
seems like that'll make maintenance and browsing a bit hard.
Any plans to break things up a bit?

{\it I have often thought about this but backed away each time, not for
  lack of time. There are just 2 main directories. One is \verb+src/+
and the other is \verb+src/AUX/+. The second contains a large number of
relatively small auxiliary functions. The first contains the driver
code. Unlike other projects which naturally lend themselves to a
hierarchical decomposition, the key principle in these utilities is very
simple. There is only one data structure, a table and only 4 kinds of
operations
\be
\item reading \(\geq 1\) column(s) from \(\geq 1\) table(s) and
producing \(\geq\) 1 column(s) in a single existing table or new one
\item Meta data operations --- show tables, describe, is sorted?
\item Reductions: sum, min, max, \ldots
\item Data Ingestion --- load, add field, \ldots
\ee

}

\item There don't seem to many tests given the number of source
files. How is the test coverage?

{\it Adequate testing is an ongoing battle! There are three basic
  approaches being taken
\bd
\item [Usage]. This consists of two approaches.
\bd
\item [TRIANGULATION] We do a lot of sanity checks as part
of regular usage of the tools. For example, a sort is often followed by
a ``head'' or ``tail'' to see whether the data is in fact sorted. 
\item [DATA INTUITION] As data scientists, we have familiarity with the
data. So, if a histogram of members by countries creates a table with more
than 250 rows, we get suspicious.
\ed
\item [Unit tests] I have started adding a number of unit tests in the
\verb+test/+ folder. Will add more in the immediate future.
\item [Probabilistic testing] This is the most ambitious effort, which
uses the formal specification to perform random probes. This has not yet
been implemented but we hope to get started in early 2014.
\ed

}
\ee


\end{document}
