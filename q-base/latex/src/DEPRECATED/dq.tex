\startreport{DQ --- Distributed Q}
\reportauthor{Ramesh Subramonian}

\section{Introduction}
\subsection{Basics of Q}
There are other documents that describe Q more completely. A brief
description is as follows.

The basic building block is a table which can be viewed as a collection
of fields or columns, each of which has the same number of cells or
values. Hence, \(T[i].f\) as the \(i^{th}\) row of column \(f\) of table
\(T\).  Every operation of Q consists of reading one or more columns
from one or more tables and producing one or more columns in an existing
table or a newly created one.

This relatively simple statement makes for great simplicity, both in
terms of implementation and programming There are a few exceptions to
this rule e.g., computing an associative operation on a column, are
the values of a column unique, meta-data based operations like listing
the tables, describing them, \ldots

\subsection{Assumptions}
\be
\item 
We assume that no data is modified in place i.e. the data read from and
written to during an operation are disjoint.
\item 
Initially, we shall ignore fault-tolerance considerations. This
assumption is relaxed in Section~\ref{fault_tolerance}.
\ee

\subsection{Terminology}
\be
\item We shall use {\bf cilkfor} to denote a loop that can benefit from
multi-threading
\item We shall use {\bf vecfor} to denote a loop that can benefit from
vectorization
\item We shall omit arguments to functions that we define when it makes
exposition simpler
\ee

\section{Global Data Structures}

A Q table consists of zero or more columns. A column is broken into
chunks of fixed number of rows (32K sounds like a reasonable
number). Chunks need not have the same size, since fields can vary in
width from bits to bytes to 8-byte doubles or integers. Note that all
fields in Q are of fixed length.  The last chunk may not be fully
utilized if the number of rows in the table is not a multiple of the
chunk size. However, it will be the same size as every other chunk.

The tables listed in Section~\ref{T_T}, ~\ref{T_D}, ~\ref{T_C} are
conceptual. We make no statement about how exactly they are
instantiated.

\subsection{Table of Tables}
\label{T_T}
\(T_T\) lists all tables in the system. Note that \(nS
\times n_C \geq n_R\). Columns are
\be
\item table name 
\item number of rows, \(n_R\)
\item chunk size, \(n_S\)
\item number of chunks , \(n_C\)
\ee

\subsection{Table of Chunk Location}
\label{T_D}
\(T_D\) tells us on which machines the chunks are located.
\be
\item table 
\item chunk index
\item location (host)
\ee

\subsection{Table of Chunks}
\label{T_C}
\(T_C\) gives us details about the chunks.  Location refers to the 
primary location.  For now, we deliberately ignore whether data is
replicated. Columns are

\be
\item table 
\item field 
\item chunk index
\item location (file)
\item size
\item number of elements
\item type \( \in \{B, I1, I2, I4, I8, F4, F8\}\)
\item does it have null values? If so, name of nn field
\item sorted? Can be ascending, descending, unsorted, unknown
\item internal or external?
\item max value
\item min value
\item sum 
\item what else? 
\ee
Meta data can be used in a variety of ways. For example
\be
\item Often, we will use a condition field, \(f_c\),  to select rows.
Then, \(min(f_c) = max(f_c) = 0\) means that none of the rows in this
chunk were selected. 
Similarly, \(min(f_c) = max(f_c) = 1\) means that all the rows in this
chunk were selected. 
\item In doing a range query or an equality query, knowing that none of
the values in this chunk are relevant, allows us to skip the chunk
without scanning data.
\item What else?
\ee

\section{Requirements of Distributed Store}

\subsection{ADDT --- Adding a table}
\label{ADDT}

Creating a table corresponds to creating one or more rows in \(T_T\) and determining
how the chunks are distributed across different nodes. For now,we assume
that the number of rows in the table is known at creation time. This
assumption is relaxed in Section~\ref{Variable_NumRows}.

\subsection{ADDF --- Adding a field}
\label{ADDF}
Creating a table corresponds to creating one or more rows in \(T_C\).
At the time of creation, \(T_C.location\) is created but the
data in it is junk. A GET after an ADDF will succeed. The developer
should bear in mind that the memory is un-initialized.

\subsection{GET}

Invocation is \(X \leftarrow GET(T, f, i) \). Returns the row in 
\(T_C[table=T, fld=f, chunk~index =i]\). In addition, we assume that
\(X\) is a \verb+void *+ pointer to memory on the local machine which
provides access to the data.

\subsubsection{Caching}

GETs are expensive if the host for that piece of memory is remote.
Hence, a local cache can avoid multiple GETs to a piece of memory that
has not changed since last requested. In subsequent discussion, we shall
assume that caching has been implemented. 

The implementation of the cache needs to take into account the details
that a cache normally has to e.g., garbage collection, invalidation,
\ldots. For example, if we delete field \(f\) from table \(T\), then the
cached copy of this data must also be deleted. Similar considerations
apply for renaming, deleting entire tables, \ldots

\subsection{ASSOC}

Invocation is \(ASSOC(T, f, v, op) \). 
Arguments are
\be
\item \(T\) --- table
\item \(f\) --- field
\item \(v\) --- value
\item \(op\) --- meta data value e.g., min, max, sum, \ldots
\ee

What this does is to perform an associative operation on
\(T_C[table=T,field=f, metadata=op].value\). This is useful when
we reduce a vector to a scalar with an associative operation such as
min, max, sum. 

\subsection{PUT}

Invocation is \(PUT(T, f, i, X, n) \). 
Arguments are
\be
\item \(X\) --- \verb+void * + pointer to local memory
\item \(n\) --- length of X
\item \(T\) --- table
\item \(f\) --- field
\item \(i\) --- chunk number
\ee

By default this is a blocking call. The asynchrononus version is
prefixed with \verb+async_+

\section{Usage Examples}
In this section, we use increasingly complex examples to see how the
primitives offered in Section~\ref{read_write} serve Q's purpose.

\subsection{Example --- one to one}
\label{eg_one_to_one}
The simplest class of operations in Q is of the form \(f_3 \leftarrow
f_1 \oplus f_2\), where all fields are in the same table, \(T\). This is
a one-to-one operation, as shown in Figure~\ref{fig_example_1}.

\begin{figure}[hb]
\begin{center}
\fbox{
  \begin{minipage}{10 cm}
\begin{tabbing} 
\hspace{0.25in} \=  \hspace{0.25in} \= 
\hspace{0.25in} \=  \hspace{0.25in} \= 
\kill
\(ADDF(T, f_3, \ldots)\) \\ 
{\bf cilkfor} \(i \leftarrow 1\) {\bf to} \(n_C\) {\bf do} \+ \\
  {\bf if} \(T_D[table=T,chunk=i].host \neq me\) {\bf then} {\bf continue} {\bf fi} \\ 
  \(X_1 \la GET(T, f_1, i) \) \\ 
  \(X_2 \la GET(T, f_2, i) \) \\ 
  \(X_3 \la GET(T, f_3, i) \) \\ 
  Let \(n \leftarrow\) chunk size \\
  {\bf vecfor} \( j \leftarrow 1\) {\bf to} \(n\) {\bf do} \+ \\ 
    \(X_3[j] \leftarrow X_1[j] \oplus X_2[j]\) \- \\
  {\bf endfor} \\ 
{\bf endfor} \\ 
{\bf wait} \\
\end{tabbing}
\caption{Figure for one-to-one Example}
\label{fig_example_1}
\end{minipage}
}
\end{center}
\end{figure}

\subsection{Example --- associative operation}
\label{eg_assoc}
Consider \(v \leftarrow \sum T.f\) as in Figure~\ref{fig_example_assoc}.

\begin{figure}[hb]
\begin{center}
\fbox{
  \begin{minipage}{10 cm}
\begin{tabbing} 
\hspace{0.25in} \=  \hspace{0.25in} \= 
\hspace{0.25in} \=  \hspace{0.25in} \= 
\kill
{\bf cilkfor} \(i_C \leftarrow 1\) {\bf to} \(n_C\) {\bf do} \+ \\
  {\bf if} \(T_D[table=T,chunk=i_C].host \neq me\) {\bf then} {\bf continue} {\bf fi} \\ 
  Initialize scalar \(v\) appropriately for operation \(\oplus\) \\ 
  \(X \la GET(T, f, i_C) \) \\ 
  Let \(n \leftarrow\) chunk size \\
  {\bf vecfor} \( j \leftarrow 1\) {\bf to} \(n\) {\bf do} \+ \\ 
    \(v \leftarrow v \oplus X[j] \) \- \\ 
  {\bf endfor} \\ 
  \(async\_ASSOC(T, f, v, op)\) \- \\ 
{\bf endfor}
\end{tabbing}
\caption{Figure for associative operation}
\label{fig_example_assoc}
\end{minipage}
}
\end{center}
\end{figure}

\subsection{Example --- aggregation}
\label{eg_agg}
We are often in the situation where we are condensing a large number of
values into a smaller number of values. For example, \(T_1\) may contain
member information, such as country, industry, \ldots and \(T_2\) may
contain all countries. We would like to count the number of members in
each country. In such cases, it is efficient to 
\be
\item replicate \(T_2\)
\item have each processor perform the computation as if only the local data
were involved 
\item accumulate the local \(T_2\) into a global \(T_2\)
\ee
Note that this is applicable not just to sum, but to other associative
operations e.g., min, max, \ldots

\begin{figure}[hb]
\begin{center}
\fbox{
  \begin{minipage}{10 cm}
\begin{tabbing} 
\hspace{0.25in} \=  \hspace{0.25in} \= 
\hspace{0.25in} \=  \hspace{0.25in} \= 
\kill
START: Make a copy of \(T_2.f_2\) in \(X_2\) \\
{\bf cilkfor} \(i \leftarrow 1\) {\bf to} \(n_C(T_2)\) {\bf do} \+ \\
  \(X_2(T_2, f_2) \la GET(T_2, f_2, i) \) \- \\ 
{\bf endfor} \\
STOP: Make a copy of \(T_2.f_2\) in \(X_2\) \\
%-------------------- 
{\bf cilkfor} \(i \leftarrow 1\) {\bf to} \(n_C(T_1)\) {\bf do} \+ \\
  {\bf if} \(T_D[table=T_2,chunk=i].host \neq me\) {\bf then} {\bf continue} {\bf fi} \\ 
  \(X_1, n \la GET(T_1, f_1, i) \) \\ 
  {\bf vecfor} \( j \leftarrow 1\) {\bf to} \(n\) {\bf do} \+ \\ 
    \(v \leftarrow v \oplus X[j] \) \- \\ 
  {\bf endfor} \\ 
  \(ASSOC(T, f, v, op)\) \- \\ 
{\bf endfor}
\end{tabbing}
\caption{Figure for aggregation}
\label{fig_example_agg}
\end{minipage}
}
\end{center}
\end{figure}

\subsection{Example --- join (sorted)}
\label{eg_join_sort}

If \(T_D\) is small, this is similar to Section~\ref{eg_agg}. In this
section, assume that it is not. 
Consider an operation of the form \(T_2.f_D \leftarrow \oplus
T_S.f_S(T_1.f_L, T_D.f_L)\). 
This is a many-to-one operation. Typically used in a join. There can be
many variations of \(\oplus\), like min, max, sum, or, and, regular,
     \ldots

\subsection{Example --- join (unsorted)}
\label{eg_join_unsort}

\TBC

\section{Strip-Mining}

Also called loop-blocking or loop-tiling. This replaces
Figure~\ref{loop_no_strip_mine} with 
Figure~\ref{loop_strip_mine}.

\begin{figure}[hb]
\begin{center}
\fbox{
  \begin{minipage}{10 cm}
\begin{tabbing} 
\hspace{0.25in} \=  \hspace{0.25in} \= 
\hspace{0.25in} \=  \hspace{0.25in} \= 
\kill
{\bf cilkfor} \(i_C \leftarrow 1\) {\bf to} \(n_C\) {\bf do} \+ \\ 
  {\bf vecfor } \( j \leftarrow 1\) {\bf to} \(n_S\) {\bf do} \+ \\ 
    \(T.f_3 \leftarrow T.f_1 \oplus T.f_2\) \-  \\
  {\bf endfor} \- \\ 
{\bf endfor} \\ 
%---------------------------
{\bf cilkfor} \(i_C \leftarrow 1\) {\bf to} \(n_C\) {\bf do} \+ \\ 
  {\bf vecfor } \( j \leftarrow 1\) {\bf to} \(n_S\) {\bf do} \+ \\ 
    \(T.f_4 \leftarrow T.f_3 \oplus T.f_5\) \- \\ 
  {\bf endfor} \- \\ 
{\bf endfor} \\ 
%---------------------------
{\bf cilkfor} \(i_C \leftarrow 1\) {\bf to} \(n_C\) {\bf do} \+ \\ 
  {\bf vecfor } \( j \leftarrow 1\) {\bf to} \(n_S\) {\bf do} \+ \\ 
    \(T.f_6 \leftarrow T.f_4 \oplus T.f_7\) \- \\ 
  {\bf endfor} \- \\ 
{\bf endfor} \\ 
\(x \leftarrow \sum T_6\) // Or some other associative function
\end{tabbing}
\caption{Loop execution without strip mining}
\label{loop_no_strip_mine}
\end{minipage}
}
\end{center}
\end{figure}

\begin{figure}[hb]
\begin{center}
\fbox{
  \begin{minipage}{10 cm}
\begin{tabbing} 
\hspace{0.25in} \=  \hspace{0.25in} \= 
\hspace{0.25in} \=  \hspace{0.25in} \= 
\kill
{\bf cilkfor} \(i_C \leftarrow 1\) {\bf to} \(n_C\) {\bf do} \+ \\ 
  {\bf vecfor } \( j \leftarrow 1\) {\bf to} \(n_S\) {\bf do} \+ \\ 
    \(T.f_3 \leftarrow T.f_1 \oplus T.f_2\)  \\
    \(T.f_4 \leftarrow T.f_3 \oplus T.f_5\) \\ 
    \(T.f_6 \leftarrow T.f_4 \oplus T.f_7\) \- \\ 
  {\bf endfor} \\ 
  \(x_{i_C} \leftarrow \sum T_6\) // Or some other associative function \- \\ 
{\bf endfor} \\ 
\(x \leftarrow \sum x_{i_C}\) // Or some other associative function \\
\end{tabbing}
\caption{Loop execution with strip mining}
\label{loop_strip_mine}
\end{minipage}
}
\end{center}
\end{figure}

\section{Fault Tolerance}
\label{fault_tolerance}

We assume the existence of a coordinator that does not go down. It is
responsible for the fault tolerance of the rest of the cluster.

\section{Speculative Computing}
\label{Speculative_Computing}

The intuition behind this idea is that, when the system is not busy,
could it be doing something that could potentially be of value later
on? Examples of this are
\be
\item finding min/max of columns, assuming these have not been computed
when the column was created. 
\ee

\section{Allowing concurrent access}

In an earlier version of Q, only one invocation of Q was allowed at any
point in time. The rationale behind this decision was that it simplified
meta-data management. Otherwise, one operation could be reading from a
field while another operation might be deleting it or renaming it or
\ldots However, it would be nice to relax this requirement. In this
section, we sketch out what this entails.

Every operation must be classified as read or write
A read operation is one that does not modify either data or meta-data.
All other operations are writes.

As usual, multiple read operations can proceed concurrently.

When a write operation starts, we lock the tables and columns being
created and release the lock at the end. What does this allow/disallow?

\be
\item If a write operation will create table \(T\), then no
other operation can 
\be
\item create, rename or read from any field in \(T\)
\item create or delete table \(T\)
\ee

\item If a write operation will create field  \(f\) in \(T\), then no
other operation can 
\be
\item delete \(T\) 
\item delete, rename or read from \(f\) 
\item create another field called \(f\) in \(T\)
\ee
However, another write operation could be creating field \(f'\) in \(T\) at the same time.
\ee

So, what happens to an operation that cannot be performed for any of the
following reasons? We have two choices and will support both using
options provided in the reqtes
\be
\item {\bf BLOCK} Operation waits until it can perform the
operation. Can it suffer starvation? \TBC
\item {\bf QUIT}  Operation does nothing and returns with appropriate error
code
\ee

\subsection{Requirements of Distributed Controller}

The distributed controller must support the following atomic operations.

\subsubsection{RESERVE}

Arguments are
\be
\item table \(T\)
\item field \(f\) (optional)
\ee

\subsubsection{FREE}

Arguments are
\be
\item table
\item field (optional)
\ee



\section{Miscellaneous Details}

\subsection{Adding fields}
\label{add_fld}

When we load data, we first ADDF and then update the data 


\subsection{Variable Number of Rows}
\label{Variable_NumRows}.

In this section, we relax the assumption that the number of rows of a
table needs to be known at run-time.

